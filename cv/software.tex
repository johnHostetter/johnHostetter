\cvsection{Selected Software Projects (\textbf{{\large {*}}} Open Source, \dag $\,$ Owner \& Sole Developer)}\vspace{-6pt}

\begin{cvpubs}
\indent
    \cvpub{\textbf{PySoft\textsuperscript{\dag} \textbf{|} 2020-2024 \textbf{|} \href{https://hostetter-lab.github.io/PySoft/}{\footnotesize \normalfont\textit{hostetter-lab.github.io/PySoft/}}:} {\footnotesize A PyTorch Soft Computing library (6.4k lines) to \textit{dynamically} self-organize neuro-fuzzy networks for supervised or reinforcement learning. Built on \texttt{igraph} and \texttt{PyTorch}, it features animated visualizations via \texttt{manim}, includes custom code for linguistic summarization or quantitative temporal transaction database mining, supporting \texttt{d3rlpy}, \texttt{sb3}, \texttt{torchrl}, \texttt{wandb}, \texttt{ClearML}, \texttt{ViZDOOM}, and so on (5.9k lines of unit tests with 94\% code coverage).}} % 6358 lines of soft, 5878 lines of unit tests, 2800 lines of experiments
    \\
    \noindent
    \cvpub{\textbf{{\large {*}}} \textbf{manim-beamer\textsuperscript{\dag} \textbf{|} 2024 \textbf{|} \href{https://pypi.org/p/manim-beamer/}{\footnotesize \normalfont\textit{pypi.org/p/manim-beamer/}}:} {\footnotesize Python emulation of LaTeX beamer within manim-slides.}}%, an interactive tool for live presentations using a math animation library called Manim (community edition).}}
    \\
    \noindent
    \cvpub{\textbf{{\large {*}}} \textbf{manim-timeline\textsuperscript{\dag} \textbf{|} 2024 \textbf{|} \href{https://pypi.org/p/manim-timeline/0.0.3/}{\footnotesize \normalfont\textit{pypi.org/p/manim-timeline}}:} {\footnotesize Seamlessly integrates history, quotes, publications, and relevant demos by gradually building upon a visual timeline for a fun, rapid and interactive literature presentation.}}
    % \\
    % \cvpub{\textbf{{\large {*}}} \textbf{Fuzzy Reinforcement Learning\textsuperscript{\dag}:} {\footnotesize Numpy code implementations for neuro-symbolic AI architectures/algorithms such as GARIC (\href{https://github.com/johnHostetter/GARIC}{https://github.com/johnHostetter/GARIC}), GPFRL (\href{https://github.com/johnHostetter/GPFRL}{https://github.com/johnHostetter/GPFRL}) and Kohonen's Self Organizing Map (\href{https://github.com/johnHostetter/Kohonen}{https://github.com/johnHostetter/Kohonen}).}}
    \\
    \cvpub{\textbf{{\large {*}}} \textbf{PolicyPrep\textsuperscript{\dag} \textbf{|} 2023-2024 \textbf{|} \href{https://github.com/johnHostetter/PolicyPrep}{\footnotesize \normalfont\textit{github.com/johnHostetter/PolicyPrep}}:} {\footnotesize A configurable ML pipeline (2,006 lines) for offline reinforcement learning in e-learning. \textbf{Saves each lab member 1 to 2 months of labor annually} with streamlined data \& policy management.}}
    \\
    \noindent
    \cvpub{\textbf{{\large {*}}} \textbf{Fuzzy Conservative Q-Learning\textsuperscript{\dag} \textbf{|} 2023 \textbf{|} \href{https://zenodo.org/records/7668308}{\footnotesize \normalfont\textit{zenodo.org/records/7668308}} \textbf{|} \href{https://github.com/johnHostetter/AAMAS-2023-FCQL}{\footnotesize \normalfont\textit{github.com/johnHostetter/AAMAS-2023-FCQL}}:} {\footnotesize %PyTorch demo of Fuzzy Conservative Q-Learning (FCQL). % using the systematic design process outlined in \href{https://dl.acm.org/doi/10.5555/3545946.3598770}{AAMAS 2023}. 
    The official PyTorch demo of the offline, model-free fuzzy reinforcement learning method published in Hostetter et al.'s AAMAS 2023.}}
    \\
    \cvpub{\textbf{{\large {*}}} \textbf{Soft-Computing\textsuperscript{\dag} \textbf{|} 2020-2022 \textbf{|} \href{https://github.com/johnHostetter/Soft-Computing}{\footnotesize \normalfont\textit{github.com/johnHostetter/Soft-Computing}}:} 
    {\footnotesize Python Numpy code for rough set theory, genetic fuzzy systems, All-Permutations Fuzzy Rule Base, Fuzzy Q-Learning, and more.}}
    \\
    \cvpub{\textbf{{\large {*}}} \textbf{Neuro-Fuzzy-Networks\textsuperscript{\dag} \textbf{|} 2019-2021 \textbf{|} \href{https://github.com/johnHostetter/Neuro-Fuzzy-Networks}{\footnotesize \normalfont\textit{github.com/johnHostetter/Neuro-Fuzzy-Networks}}:} {\footnotesize Numpy code of adaptive neuro-fuzzy networks (HyFIS \& SaFIN), and more related to neuro-symbolic AI (e.g., \href{https://github.com/johnHostetter/GPFRL}{GPFRL}, \href{https://github.com/johnHostetter/GARIC}{GARIC}, and \href{https://github.com/johnHostetter/Kohonen}{Kohonen}).}}
    \\
    \noindent
    \cvpub{\textbf{HepiusApp (formerly OsteoApp) \textbf{|} 2019 \textbf{|} \href{https://ronak1997.github.io/Hepius/}{\footnotesize \normalfont\textit{ronak1997.github.io/Hepius/}}:} {\footnotesize A Xamarin cross-platform mobile app to research smartphone use in healthcare decisions related to Osteoporosis. \textbf{Transferred intellectual property to Penn State Research Foundation.} Supervised clinical trials led by Dr. Russell Kirkscey, Dr. Edward Fox, and Dr. Hien Nguyen. Resulted and featured in 2 publications:}} \vspace{-16pt}
    \begin{enumerate}
        \setlength\itemsep{-0.5em}
        \item     \href{https://pure.psu.edu/en/publications/development-and-patient-user-experience-evaluation-of-an-mhealth-}{{\texttt{Kirkscey, R. (2021). Development and Patient User Experience Evaluation of an mHealth Informational\\App for Osteoporosis. International Journal of Human–Computer Interaction, 38(8), 707–718. 
    }}}
    \item     \href{https://pure.psu.edu/en/publications/mhealth-apps-for-older-adults-a-method-for-development-and-user-e}{{\texttt{Kirkscey, R. (2021). mHealth Apps for Older Adults: A Method for Development and User Experience\\Design Evaluation. Journal of Technical Writing and Communication, 51(2), 199-217. 
    %https://doi.org/10.1177/0047281620907939
    }}}
    \end{enumerate}
\end{cvpubs}

