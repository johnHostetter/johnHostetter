%-------------------------
% Rover Resume - Base Template
% Link: https://github.com/subidit/rover-resume
%
% Shows code for various formatting options for different resume sections.
% Education and Projects have single-line headers; while Experience uses double-line.
% Some formatting codes are kept inline; consider \newcommand{cmd}{def}.
% Excludes hyperref and icons for readability; MVP version.
% Explore other templates for more options.
% Mix and match as desired. Be consistent with headers and sub-headers.
%------------------------

\documentclass[11pt]{article} % fontsize 10pt/11pt/12pt

\usepackage[margin=0.55in, a4paper]{geometry}
\setcounter{secnumdepth}{0} % remove section numbering
\usepackage{titlesec}
\titlespacing{\subsection}{0pt}{*0}{*0} % remove vertical spacing above and below
\titlespacing{\subsubsection}{0pt}{*0}{*0}
\titleformat{\section}{\small\bfseries\uppercase}{}{}{}[\titlerule]
\titleformat*{\subsubsection}{\small\itshape}
\usepackage{enumitem}
\usepackage{hyperref}  % added by me (John)
% \setlist[itemize]{noitemsep,left=0pt .. \parindent}
\pagestyle{empty} % remove page number
% \pdfgentounicode=1


\begin{document}

\begin{center}
	\begin{minipage}{0.5\textwidth}
		{\LARGE\bfseries
			John Wesley Hostetter % Name Here
		} \\ \medskip
		Computer Science Ph.D. Candidate % Title [optional]
	\end{minipage} \hfill
	\begin{minipage}{0.475\textwidth}
		\raggedleft
        \small
		% Mobile: 717-228-8891  \\
		LinkedIn: \href{https://www.linkedin.com/in/john-w-hostetter/}{linkedin.com/in/john-w-hostetter} \\
		GitHub: \href{https://github.com/johnHostetter}{github.com/johnHostetter} \\
		Email: jwhostet@ncsu.edu
    \end{minipage}
\end{center}

\vspace{-24pt}

\section{Experience}
\vspace{-8pt}

%=================%
\noindent\textbf{Graduate Research Assistant \hfill Aug 2021 - Present}
\subsubsection{North Carolina State University \hfill  Raleigh, NC \& Remote}
\vspace{-2pt}
\begin{itemize}
\setlength\itemsep{-0.5em}
	\item First to achieve offline, model-free fuzzy RL; published in \href{https://dl.acm.org/doi/10.5555/3545946.3598770}{AAMAS 2023} (A+ CORE Rank)
     \item Oversaw setup of 18 AIs across 8 IRB studies involving a total of 2,770 human-subjects
    \item \href{https://scholar.google.com/citations?user=lJkPFpwAAAAJ&hl=en}{Publications:} EDM, UMUAI, IVA {\footnotesize (Best Paper Finalist)}, FUZZ {\footnotesize (x2)}, AIED {\footnotesize (x2)}, CogSci {\footnotesize (x2)}
    % \item Developed an ML pipeline (21k lines) saving each member 1-2 months of labor annually
    \item Awarded SIGAI Student Travel Grant \& AAMAS Student Scholarship 
    % \item Best paper finalist (x2); awarded SIGAI Travel Grant \& AAMAS Student Scholarship  % SIGAI Student Travel Grant
\end{itemize}

\vspace{-4pt}

\noindent\textbf{Assistant Instructor \hfill July 2022 - July 2024}
\subsubsection{North Carolina State University's \href{https://ai-academy.ncsu.edu/}{Artificial Intelligence Academy} \hfill Remote}
\vspace{-2pt}
\begin{itemize}
	\item Taught $\sim$1,000 industry experts (e.g., NASA, Lexmark, JPMorgan Chase \& Co.)
\end{itemize}

\vspace{-4pt}

\noindent\textbf{Graduate Teaching \& Service Assistant \hfill Aug 2019 - Dec 2021}
\subsubsection{North Carolina State University \hfill Raleigh, NC \& Remote}
\vspace{-2pt}
\begin{itemize}
\setlength\itemsep{-0.5em}
	\item Assist with 485 students (total) in: {\footnotesize $[$CSC 422/522$]$} Automated Learning Data Analysis (x2), {\footnotesize $[$CSC 333$]$} Automata, Grammars \& Computability, {\footnotesize $[$CSC 216$]$} Programming Concepts - Java
    \item Create coursework for NC State AI Academy's Data Science, CSC 422/522 and CSC 333
\end{itemize}

\vspace{-4pt}

\noindent\textbf{Application Developer Intern \hfill June 2018 - Aug 2018}
\subsubsection{Global Data Consultants IT Solutions \hfill Mechanicsburg, PA}
\vspace{-2pt}
\begin{itemize}
\setlength\itemsep{-0.5em}
	\item Xamarin certified professional; built an app for clients to approve contractors' timesheets
\end{itemize}

\vspace{-20pt}
\section{Education}
\vspace{-8pt}

%=================%
\noindent\textbf{North Carolina State University $|$ {\normalfont\itshape Computer Science M.Sc. \& Ph.D.} \hfill 2019-2025}
\vspace{-6pt}
\begin{itemize}
\setlength\itemsep{-0.5em}
    \item \textit{Volunteer Student Coordinator} for Doctoral Recruiting (2023 \& 2024 cohorts)
    \item \textit{Peer Reviewer} for ICLR 2025, NeurIPS 2024, EAAI 2023 \& 2024, CHI 2022 \& 2024
    \item Cumul. GPA: 3.83 w/ Graduate Merit Award (x3) \& Student Travel Grant (x2)
\end{itemize}

\vspace{-4pt}

\noindent\textbf{The Pennsylvania State University $|$ {\normalfont\itshape Computer Science B.Sc.} \hfill 2016-2019}
\vspace{-6pt}
\begin{itemize}
    \setlength\itemsep{-0.5em}
    \item \textit{Secretary} for Association for Computing Machinery (ACM)
    \item Cumul. GPA: 3.66 \& Dean's List (x4)
\end{itemize}

\vspace{-4pt}
\noindent
Engineering Physics coursework at \textbf{Embry-Riddle Aeronautical University} from 2015-2016

% \vspace{-12pt}

% \section{Certification \& Awards}
% %===============================%
% \begin{enumerate}
% 	\item Some programming bootcamp, Location. \hfill 2023
% 	\item Forbes Top 15 Procrastinators under 15. \hfill 2022
% 	\item Perticipation award for attending workshop. \hfill 2021
% \end{enumerate}

\vspace{-12pt}

\section{Selected Software (\dag $\,$ Sole Owner) {\normalfont\footnotesize $[$Name, Tech(s), Code Cov. (opt'l), URL$]$}}
\vspace{-8pt}
%=================
\noindent\textbf{fuzzy-theory\textsuperscript{\dag} $|$ \texttt{PyTorch, igraph} $|$ {\normalfont3.6k LOC w/ 93\% cov.} $|$ \href{https://pypi.org/p/fuzzy-theory}
{\normalfont\textit{pypi.org/p/fuzzy-theory}} \hfill 2024}
\vspace{-6pt}
\begin{itemize}
\setlength\itemsep{-0.5em}
  \item Transparent approximate reasoning via neuro-fuzzy networks, fuzzy sets \& fuzzy logic operations
\end{itemize}

\vspace{-4pt}

\noindent\textbf{rough-theory\textsuperscript{\dag} $|$ \texttt{igraph} $|$ {\normalfont1.8k LOC w/ 85\% cov.} $|$ \href{https://pypi.org/p/rough-theory}
{\normalfont\textit{pypi.org/p/rough-theory}} \hfill 2024}
\vspace{-6pt}
\begin{itemize}
\setlength\itemsep{-0.5em}
  \item Leverage discernibility to identify core knowledge (e.g., cut provably irrelevant attributes)
\end{itemize}

\vspace{-4pt}

\noindent\textbf{fuzzy-ml\textsuperscript{\dag} $|$ \texttt{igraph} $|$ {\normalfont3.3k LOC w/ 75\% cov.} $|$ \href{https://pypi.org/p/fuzzy-ml}
{\normalfont\textit{pypi.org/p/fuzzy-ml}} \hfill 2024}
\vspace{-6pt}
\begin{itemize}
\setlength\itemsep{-0.5em}
  \item Fuzzy clustering, linguistic summaries and association analysis of temporal quantitative databases
\end{itemize}

\vspace{-4pt}

\noindent\textbf{regime\textsuperscript{\dag} $|$ \texttt{igraph} $|$ {\normalfont 754 LOC w/ 99\% cov.} $|$ \href{https://pypi.org/p/regime}
{\normalfont\textit{pypi.org/p/regime}} \hfill 2024}
\vspace{-6pt}
\begin{itemize}
\setlength\itemsep{-0.5em}
  \item Lightweight MLOps; inspect, visualize, validate workflows \& hyperparameters
\end{itemize}

\vspace{-4pt}

\noindent\textbf{manim-timeline\textsuperscript{\dag} $|$ \texttt{manim-slides, igraph} $|$ \href{https://pypi.org/p/manim-timeline}
{\normalfont\textit{pypi.org/p/manim-timeline}} \hfill 2024}
\vspace{-6pt}
\begin{itemize}
\setlength\itemsep{-0.5em}
  % \item Seamlessly integrate history (e.g., quotes) on an elegant timeline for rapid presentation
  \item A seamless and elegant timeline for rapid presentation of related literature
\end{itemize}

\vspace{-4pt}

\noindent\textbf{manim-beamer\textsuperscript{\dag} $|$ \texttt{manim-slides} $|$ \href{https://pypi.org/p/manim-beamer}
{\normalfont\textit{pypi.org/p/manim-beamer}} \hfill 2024}
\vspace{-6pt}
\begin{itemize}
\setlength\itemsep{-0.5em}
  \item Emulate \LaTeX $\:$ beamer in Python to animate technical and professional slides
\end{itemize}

\vspace{-4pt}

\noindent\textbf{PySoft\textsuperscript{\dag} $|$ \texttt{PyTorch} $|$ {\normalfont 3.1k LOC w/ 94\% cov.} $|$ \href{https://hostetter-lab.github.io/PySoft}{\normalfont\textit{hostetter-lab.github.io/PySoft}} \hfill 2020-2024}
\vspace{-6pt}
\begin{itemize}
\setlength\itemsep{-0.5em}
  \item Reinforcement \& supervised learning of neuro-symbolic networks
  \item Supports d3rlpy, sb3, skorch, rpy2, sympy, wandb, etc. for rapid prototyping
  % \item Produced at least 3 publications (AAMAS 2023, FUZZ IEEE $[$x2$]$)
\end{itemize}

\vspace{-4pt}

% \cvpub{\textbf{PolicyPrep\textsuperscript{\dag} \textit{(open-source)}:} {\footnotesize A configurable pipeline automates experiment setup for offline reinforcement learning studies in education. It streamlines data management, policy induction, evaluation, and saves each lab member 1 to 2 months of labor annually. \href{https://github.com/johnHostetter/PolicyPrep}{https://github.com/johnHostetter/PolicyPrep}}}
%     \\

\noindent\textbf{PolicyPrep\textsuperscript{\dag} $|$ \texttt{PyTorch, d3rlpy} $|$ {\normalfont 3.9k LOC} $|$ \href{https://github.com/johnHostetter/PolicyPrep}{\normalfont\textit{github.com/johnHostetter/PolicyPrep}} \hfill 2023-2024}
\vspace{-6pt}
\begin{itemize}
\setlength\itemsep{-0.5em}
  \item Configurable MLOps for offline reinforcement learning in e-learning; saves $\sim$1-2 months annually
  % \item Includes rough set theory, genetic fuzzy systems, neuro-fuzzy networks, etc.
\end{itemize}

\vspace{-4pt}

\noindent\textbf{Fuzzy Conservative Q-Learning Demo\textsuperscript{\dag} $|$ \texttt{PyTorch} $|$ \href{https://zenodo.org/records/7668308}{\normalfont\textit{zenodo.org/records/7668308}} \hfill 2023}
\vspace{-6pt}
\begin{itemize}
\setlength\itemsep{-0.5em}
    \item A self-organizing, transparent neuro-fuzzy network; published in \href{https://dl.acm.org/doi/10.5555/3545946.3598770}{AAMAS 2023}
  % \item Available on \href{https://github.com/johnHostetter/AAMAS-2023-FCQL}{GitHub} and \href{https://zenodo.org/records/7668308}{Zenodo}: \texttt{johnHostetter/AAMAS-2023-FCQL:} \texttt{First release (v1.0.0). Zenodo. https://doi.org/10.5281/zenodo.7668308}
  % \item Includes rough set theory, genetic fuzzy systems, neuro-fuzzy networks, etc.
\end{itemize}

\vspace{-4pt}

\noindent\textbf{Neuro-Symbolic AI\textsuperscript{\dag} $|$ \texttt{Numpy} $|$ \href{https://github.com/johnHostetter/Soft-Computing}{\normalfont\textit{github.com/johnHostetter/Soft-Computing}} \hfill 2020-2022}
\vspace{-6pt}
\begin{itemize}
\setlength\itemsep{-0.5em}
  \item Neuro-symbolic AIs (e.g., genetic fuzzy systems) and soft computing operations
  % \item Includes rough set theory, genetic fuzzy systems, neuro-fuzzy networks, etc.
\end{itemize}

% \vspace{-6pt}
% \noindent
% and at least 7 others (e.g., \href{https://github.com/johnHostetter/GPFRL}{GPFRL}, \href{https://github.com/johnHostetter/GARIC}{GARIC} \href{https://github.com/johnHostetter/manim-beamer}{manim-beamer}, \href{https://github.com/johnHostetter/PolicyPrep}{PolicyPrep}, \href{https://github.com/johnHostetter/AAMAS-2023-FCQL}{FCQL})

% \vspace{-12pt}
\vspace{-4pt}

% \section{Selected Closed-Source Software}
% \vspace{-8pt}
%=================

\noindent\textbf{HepiusApp (formerly OsteoApp) $|$ \texttt{C\#, Xamarin} $|$ \href{https://ronak1997.github.io/Hepius/}{\normalfont\textit{ronak1997.github.io/Hepius/}} \hfill 2019}
\vspace{-6pt}
\begin{itemize}
\setlength\itemsep{-0.5em}
  \item Built a mobile app to research smartphone use in healthcare decisions for Osteoporosis
  \item Transferred IP to \href{https://ott.psu.edu/penn-state-research-foundation/}{Penn State Research Fdn.}; articles in \href{https://pure.psu.edu/en/publications/development-and-patient-user-experience-evaluation-of-an-mhealth-}{Intl. Journal of HCI} \& \href{https://pure.psu.edu/en/publications/mhealth-apps-for-older-adults-a-method-for-development-and-user-e}{JTWC} %in 2019
\end{itemize}
\vspace{-4pt}
\noindent
and at least 4 more related to neuro-symbolic AI (e.g., \href{https://github.com/johnHostetter/GPFRL}{GPFRL}, \href{https://github.com/johnHostetter/GARIC}{GARIC}, \href{https://github.com/johnHostetter/Kohonen}{Kohonen}, \href{https://github.com/johnHostetter/Neuro-Fuzzy-Networks}{SaFIN})

\vspace{-12pt}

\section{Technical Skills}
\vspace{-8pt}
%===========================%
\begin{description}[itemsep=0pt]
\setlength\itemsep{-0.5em}
    \item[Interests:] Fuzzy Logic, Neuro-Symbolic AI, Rough Sets, eXplainable AI, Machine Learning
    \item[Languages:] \textit{Python (current)}, Java, C\#, C++, C, R, Scheme, JavaScript, SQL
	\item[Software:] Git, \LaTeX, IBM SPSS Statistics, Blender, Inkscape, Adobe XD
    \item[Operating Systems:] Ubuntu 22.04.4 LTS, Fedora 36, Windows 7-11
\end{description}

\end{document}